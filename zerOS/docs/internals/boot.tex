\section[Overview]{Boot process overview}

Before the kernel can jump to the "main" code, some setup code is required. For example, since we may want to use some processor-specific instructions (such as SSE extensions, or AVX ones), we need to enable them.

We also need to setup a proper stack, GDT, IDT, etc\dots We also initialize important parts of zerOS, such as its memory manager, and its scheduler.

Hence, this chapter will detail the boot process of zerOS, from the bootloader to the kernel.

\section{Bootloaders}

zerOS will support different bootloaders, wich are detailed in the following sections. At the time of writing, only \hyperref[sec:limine]{Limine} is supported.

\subsection{Multiboot2}

\FIXME{Multiboot2 support is not implemented yet.}

\subsection{Limine}

\label{sec:limine}

Although Limine allows to boot on a lot of different firmware/architecture combinations, only UEFI/x86-64 is supported by zerOS, for now.

\TODO{Provide a list of used Limine bootloader requests, why we need them, and how they are handled by zerOS.}

\subsection{UEFI boot stub}

\FIXME{UEFI boot stub support is not implemented yet.}

\subsection{µloader}

\FIXME{µloader support is not implemented yet.}